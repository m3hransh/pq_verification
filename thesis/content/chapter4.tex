\chapter { Priority Queue Implementations }
\section { Specification of Priority Queue Properties }

Priority queues or heaps are data structures that provide efficient access to the high priority element (often minimum element).
that support the following operations:

\begin{itemize}
	\item \textbf{Insert}: Add an element with a given priority.
	\item \textbf{Find-Min}: Retrieve the element with the highest priority (lowest value).
	\item \textbf{Delete-Min}: Remove the element with the highest priority.
	\item \textbf{Merge}: Combine two priority queues into one.
\end{itemize}

For the sake of simplicity, we will focus on the \textbf{min-heap} variant, where the element with the lowest value has the highest priority.

\section { Leftist Heap Implementation }
Haskell code, explanation, and verification.

\section { Binary Heap Implementation }
Code structure, key invariants, and proof techniques.

\section { Skew/Binomial/Pairing Heap (optional) }
More advanced structure and proof techniques (if time allows).

\section { Comparison of Implementations }
Performance, expressiveness, and verification effort.
