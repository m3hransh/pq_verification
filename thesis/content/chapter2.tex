\chapter { Background and Related Work }
\section { Functional Data Structures }
Functional data structures are \emph{immutable}, meaning their state cannot be changed after creation, and \emph{persistent}, allowing access to previous versions of the structure.
Combined with recursive algebraic data types (ADTs), this enables efficient and elegant implementations that are often easier to reason about compared to their imperative counterparts~\cite{okasaki}.

In contrast, imperative and mutable data structures permit in-place modifications, which can introduce side effects such as data races or unintended state changes in concurrent environments.
By ensuring that each operation produces a new version of the structure without altering the original, functional data structures provide strong guarantees of referential transparency and purity.
These properties not only improve modularity and composability but also facilitate formal reasoning and verification, since invariants are preserved across all versions of the structure~\cite{okasaki}.

This thesis focuses on functional data structures, specifically priority queues, due to their suitability for formal verification and the rich body of existing research in this area.

\section { Program Verification Techniques }
Overview of Hoare logic, model checking, interactive theorem proving, etc.
\todo{Finish section program verification techniques}

\section { Related Work }
\todo{Finish section related work}
Comparison with Coq, Agda, Dafny, or other tools verifying similar structures.

