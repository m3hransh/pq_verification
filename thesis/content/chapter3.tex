\chapter{Priority Queue Implementations}
\label{ch:implementations}
\section{Specification of Priority Queue Interface}

Priority queues are multisets with an associated priority for each element, allowing efficient retrieval of the element with the highest (or lowest) priority.
To avoid confusion with FIFO queues, we will refer to them as "heaps" throughout this thesis.

Typical operations include:
\begin{itemize}
	\item \textbf{insert}: Add a new element with a given priority.
	\item \textbf{findMin}: Retrieve the element with the minimum key (in the min-heap variant).
	\item \textbf{splitMin}: Return a pair consisting of the minimum key and a heap with that minimum element removed.
	\item \textbf{merge}: Combine two priority queues into one.
\end{itemize}

Below is the specification of the priority queue interface, defined as a Haskell type class.
\texttt{MinView} is a utility type to represent the result of the \texttt{splitMin} operation,
which returns the minimum element and the remaining heap.

\begin{code}[caption={Leftist Heap Implementation in Haskell}, label={lst:pq}]

	data MinView q a =
	EmptyView | Min {minValue :: a, restHeap :: q a}
	deriving (Show, Eq)

	class PriorityQueue pq where
	empty :: (Ord a) => pq a
	isEmpty :: (Ord a) => pq a -> Bool
	findMin :: (Ord a) => pq a -> Maybe a
	insert :: (Ord a) => a -> pq a -> pq a
	splitMin :: (Ord a) => pq a -> MinView pq a
\end{code}

Priority queues are widely used in computer science and engineering.
They play a central role in \emph{operating systems} for task scheduling, in \emph{graph algorithms} such as Dijkstra’s shortest path and Prim’s minimum spanning tree, and in \emph{discrete event simulation}, where events are processed in order of occurrence time.
Other applications include data compression (e.g., Huffman coding) and networking (packet scheduling).

In this thesis, we focus on the \emph{min-priority queue}, where elements with lower keys are considered higher priority.
We will study and verify functional implementations of \emph{Leftist Heaps} priority queues.

% , \emph{Binary Heaps}, and an optional third variant such as Skew Heaps or Binomial Heaps %TODO: Might not be covered
% Each implementation offers different trade-offs in terms of complexity, ease of reasoning, and verification effort.

\section{Leftist Heap Implementation}

Leftist heaps, introduced by Crane~\cite{crane1972} and discussed extensively by
Knuth~\cite{knuth1973taocp3}, are a variant of binary heaps designed to support
efficient merging. They are defined by two key invariants:

\begin{itemize}
	\item \textbf{Heap Property} – For every node, the stored key is less than or equal
	      to the keys of its children. This ensures that the minimum element is always
	      found at the root.

	\item \textbf{Leftist Property} – For every node, the rank (also called the
	      \emph{right spine length}, i.e., the length of the rightmost path from the node in question to an empty node)
	      of the left child is greater than or equal to that of the right child.
	      This property ensures that the right spine of the heap is kept as short as
	      possible, which in turn guarantees logarithmic time complexity for merging
	      operations\cite{okasaki}.
\end{itemize}

We represent leftist heaps using a recursive algebraic data type in Haskell, as described by Okasaki~\cite{okasaki}:
\begin{code}[caption={Leftist Heap data type}, label={lst:leftist}]
	data LeftistHeap a
	= EmptyHeap
	| HeapNode
		{ value :: a
			, left  :: LeftistHeap a
			, right :: LeftistHeap a
			, rank  :: Int
		}
\end{code}

Each node contains a value, its left subtree, right subtree, and its rank.

The merge operation merges the right subtree of the heap with the smaller root value with the other heap. After merging,
it adjusts the rank by swapping the left and right subtrees if necessary using the function \texttt{makeHeapNode}.
\begin{code}[caption={Leftist Heap merge}, label={lst:leftist-merge}]

	heapMerge :: (Ord a) => LeftistHeap a -> LeftistHeap a -> LeftistHeap a
	heapMerge EmptyHeap EmptyHeap = EmptyHeap
	heapMerge EmptyHeap h2@(HeapNode _ _ _ _) = h2
	heapMerge h1@(HeapNode _ _ _ _) EmptyHeap = h1
	heapMerge h1@(HeapNode x1 l1 r1 _) h2@(HeapNode x2 l2 r2 _)
	| x1 <= x2 = makeHeapNode x1 l1 (heapMerge r1 h2)
	| otherwise = makeHeapNode x2 l2 (heapMerge h1 r2)
\end{code}
Because the the right spine is kept short by the leftist property and at most is logarithmic, the merge operation runs in \(O(\log n)\) time.
\begin{code}[caption={Leftist Heap helper functions}, label={lst:leftist-helpers}]

	makeHeapNode :: a -> LeftistHeap a -> LeftistHeap a -> LeftistHeap a
	makeHeapNode x h1 h2
	| rrank h1 >= rrank h2 = HeapNode x h1 h2 (rrank h2 + 1)
	| otherwise = HeapNode x h2 h1 (rrank h1 + 1)
\end{code}

Other functions are straightforward to implement.
\begin{code}[]
	heapEmpty :: (Ord a) => LeftistHeap a
	heapEmpty = EmptyHeap

	heapFindMin :: (Ord a) => LeftistHeap a -> Maybe a
	heapFindMin EmptyHeap = Nothing
	heapFindMin (HeapNode x _ _ _) = Just x

	heapIsEmpty :: (Ord a) => LeftistHeap a -> Bool
	heapIsEmpty EmptyHeap = True
	heapIsEmpty _ = False

	heapInsert :: (Ord a) => a -> LeftistHeap a -> LeftistHeap a
	heapInsert x h = heapMerge (HeapNode x EmptyHeap EmptyHeap 1) h

	heapSplit :: (Ord a) => LeftistHeap a -> MinView LeftistHeap a
	heapSplit EmptyHeap = EmptyView
	heapSplit (HeapNode x l r _) = Min x (heapMerge l r)
\end{code}

In the chapter \ref{ch:verification}, we will verify that these implementations satisfy the priority queue interface and maintain the leftist heap invariants.

